\section{Motivation}

Ein Transportüberwachungssystem kann aus mehreren Blickwinkeln betrachtet werden.
	In diesem Semesterprojekt war für die Erstellung eines vollständigen
	Systems zu wenig Zeit vorhanden und somit wurden in diesem Projekt der Fokus
	auf die Erstellung einer mobilen Applikation, der Replizierung von Daten auf
	ein Backendserver sowie die Verwaltung des Systems mit einer Webapplikation
	gelegt.
	
Aus erwähnten zeitlichen Gründen wurde keine eigene Hardware entwickelt und somit
	die Android Plattform als Host-System für eine mobile Applikation verwendet.
	Zudem wurden bis auf eine Ausnahme keine realen Sensoren verwendet sondern
	benötigte Sensoren simuliert.

Dadurch hat sich diese Projekt auf die Entwicklung von Software konzentriert.
	Wir haben uns in neuen Technologien, teilweise sogar in Beta-Versionen,
	eingearbeitet und diese exzessiv in diesem Projekt verwendet.
	
Auf dem Backendserver wurde mit CouchDB ein unkonventioneller Ansatz der
	Datenpersistierung gewählt. Zur Verwaltung von Aufträgen (Lieferungen)
	wurde eine Webapplikation erstellt.

Als Softwareentwicklungsprozess wurde \emph{Test-Driven-Development} gewählt.
	Hierzu wurden nahezu alle entwickelten Komponenten mit \emph{Unit-Tests}
	getestet sowie wenn möglich auf einem \emph{Continius-Integration}-Server
	bei jeder Änderung automatisiert getestet.
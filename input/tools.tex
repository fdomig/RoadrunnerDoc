\section{Projektunterstützende Werkzeuge und Hilfsmittel}
\label{sec:tools}

In diesem Projekt werden mehrere projektunterstützende Werkzeuge verwendet.
	Durch den sehr agilen Entwicklungsprozess (siehe Kapitel~\ref{sec:development})
	wird vor allem auf eine entsprechende Versionskontrolle sowie ein
	Continuous-Integration-Server zur Überwachung des jeweiligen Entwicklungsstands
	geachtet. Diese beiden Systeme sind in den nachfolgenden Abschnitten erläutert.

\subsection{Versionskontrollsystem GIT mit github.com}
\label{subsec:git}

Git ist ein verteiltes Revisions-Kontroll-System, dessen Schwerpunkt auf
	Geschwindigkeit liegt. Git wurde ursprünglich von Linus Torvalds für
	Linux-Kernel-Entwicklung \cite{Torvalds07} entworfen und entwickelt.
	Jedes Git-Arbeitsverzeichnis ist ein vollwertiges \emph{Repository} mit
	kompletter Historie und vollständigem Commit-Tracking.
	Git ist nicht abhängig von einem Netzwerkzugang oder einen zentralen
	Server, das heißt es kann auch nur lokal entwickelt werden.
	Git ist freie Software unter der GPL2 (GNU General Public License
	Version 2) verteilt.
	
In diesem Projekt wird Git für alle Teile des Projekts eingesetzt. Es wird sowohl
	die CouchDB Applikation, die Android Applikation, die Webapplikation sowie die
	Dokumentation via Git verwaltetet. Alle Projektmitglieder können alle
	Komponenten einsehen und bearbeiten.
	
Um Git auch Online zu synchronisieren, wird in diesem Projekt
	gihub.com\footnote{vgl. \url{http://github.com}} verwendet. Via github.com ist es
	möglich, zusätzliche projektunterstützende Werkzeuge (wie z.B. Issue-Tracking,
	Wiki, etc.) zu verwenden.

\subsection{Continuous Integration mit Jenkins}
\label{subsec:ci}

Jenkins ist eine Open-Source-Continuous-Integration (CI)-Tool, welches in Java
	geschrieben ist. Jenkins bietet kontinuierliche Integrations-Services für
	Software-Entwicklung an. Es ist ein server-basiertes System
	mit einem Servlet-Container wie Apache Tomcat. Es unterstützt SCM-Tools
	wie CVS, Subversion, Git und Clearcase. Zudem werden die Build-Tools
	Apache Ant und Apache-Maven, sowie beliebige Shell-Skripte und
	Windows-Batch-Befehle unterstützt. Jenkins ist freie Software und wird
	unter der MIT-Lizenz (Massachusetts Institute of Technology License) veröffentlicht.
	
In diesem Projekt wird sowohl die Android Applikation (Java) sowie die
	Webapplikation (PHP) automatisiert via Jenkins getestet. Bei jedem neuen
	Commit, welcher zu github.com synchronisiert wird, wird automatisiert
	ein Checkout von Jenkins in ein neues Verzeichnis erstellt. Danach
	wird das jeweilige Ant-Build-Script $build.xml$ ausgeführt und die
	Software kompiliert. Anschließend werden die definierten
	Analysewerkzeuge mit der jeweiligen Konfiguration ausgeführt.
	
Bei der Android Applikation werden JUnit\footnote{vgl.
	\url{http://www.junit.org/}} Tests ausgeführt.
	
Bei der Webapplikation werden PHPUnit\footnote{vgl.
	\url{http://www.phpunit.de/}}, PHP-Check-Style, PHP-Mess-Detector sowie
	PHP-Copy-Paste-Detector verwendet um die Qualität zu überprüfen.
	


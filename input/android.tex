\section{Android Applikation}
\label{sec:android}

Als Hauptsystem in diesem Projekt wird eine Applikation für die Android
	Plattform erstellt. Diese Applikation dient der mobilen Überwachung
	von Lieferungen bzw. den Gegenständen einer Lieferung.

Android\footnote{vgl. \url{http://www.android.com/}} ist ein Betriebssystem sowie auch
	eine Software Plattform für mobile Geräte. Es werden Smartphones, Netbooks,
	Mobiltelefone und Tablets unterstützt. Entwickelt wurde das Betriebssystem
	von der \emph{Open Handset Alliance}\footnote{vgl.
	\url{http://www.openhandsetalliance.com/}},
	einem Konsortium von 80 Firmen zur Schaffung offener Standards für mobile Geräte,
	die von Google im Jahr 2007 gegründet wurde (vgl. \cite{OHA07}). 

In diesem Projekt wird Android gewählt, da es die gewünschten Anforderungen an unsere
	mobile Applikation am besten erfüllt. Ein wichtiger Aspekt für diese Entscheidung
	ist, dass Android Open-Source ist und ständig verbessert wird.

Android unterstützt ebenfalls CouchDB, welche auch in unserem Backend System einsetzen,
	was ein weiters wichtiges Kriterium ist. Das Android SDK ist als Plugin für
	Eclipse\footnote{vgl. \url{http://www.eclipse.org/}} verfügbar und ermöglichte
	es uns somit ein plattformunabhängige Entwicklung in einer gewohnten
	Entwicklungsumgebung. Ein weiterer Vorteil von Android ist, dass die Smartphones
	von mehreren verschiedenen Herstellern angeboten werden, was dem Kunden einen
	Gewissen Spielraum bei der Anschaffung der Smartphones gibt. Ebenfalls ist
	ein einfaches Deployment, sowie Updates für die Anwendung wichtig. Dies wird mit
	$adb$ oder App-Installer ermöglicht. Ein Nachteil könnte die fehlerhafte
	Bedienung des Benutzers sein, die die Funktionsweise der mobilen Applikation
	beeinträchtigten könnte.

\subsection{Benutzung}

Nach dem sich ein Benutzer mit seinem Benutzernamen, seinem Passwort und der ausgewählten Transporteinheit angemeldet hat, gelangt er auf den Home Screen.
Die Benutzeroberfläche des Home Screens ist einfach und intuitiv gestaltet. Der Benutzer hat dann die Möglichkeit 
Barcodes zu scannen sowie die Lieferdetails zu den aktuell geladenen Paketen anzusehen.
Nachdem der Barcode eines Pakets gescannt wurde, kann abhängig vom Status des Paketes (aufgeladen - nicht aufgeladen), das Paket
geladen, entladen oder abgeliefert werden. Wenn ein Paket abgeliefert wird, kann der Kunde die Lieferung mit seiner Unterschrift bestätigen.

Mit einem Klick auf den Menüpunkt \emph{My Deliveries} sieht der Benutzer die Adressinformation des Senders bzw. des Empfängers für jede Lieferung.
Weiters können alle Pakete und deren Status einer ausgewählten Lieferung angezeigt werden. 
Außerdem kann die Karte mit der eingezeichneten Route und der aktuellen Positionen des Benutzers angezeigt werden.

\subsection{Implementierung}

Bei der Implementierung der Benutzeroberfläche werden bekannte \emph{UI-Patterns} verwendet. Auf dem Home Screen der Applikation
wird das \emph{Dashboard Pattern} verwendet um die Navigation in der Applikation klar und einfach zu gestalten. Des Weiteren enthält jeder
Screen eine \emph{Action Bar} im oberen Teil, die dem Benutzer anzeigt in welchem Kontext der Applikation er sich gerade befindet (\emph{Breadcrumb Navigation}).
Mit einem Klick auf das Home Symbol in der Action Bar hat der Benutzer jederzeit die Möglichkeit wieder auf den Home Screen zu gelangen.

Die Dienste zur Überwachung der Temperatur und der GPS-Position sowie zum replizieren der Daten mit der serverseitigen Datenbank sind als \emph{Android Services},
die im Hintergrund laufen, implementiert. Somit wird gewährleistet, dass diese wichtigen Dienste unabhängig von der Applikation, die auch vom Benutzer geschlossen werden kann,
durchgeführt werden.

\subsection{Mögliche Erweiterungen}

Mögliche Erweiterungen, die in diesem Projekt nicht umgesetzt werden, wären beispielsweise eine automatische Berechnung der besten Route, abhängig
von den zu erledigenden Lieferungen. Weiters könnte die berechnete Route in das Navigationssystem Navigation von Google übernommen werden.
Ein weiterer Nutzen für den Benutzer wäre auch die Darstellung der aktuellen Temperaturwerte seiner geladenen Güter, sowie eine Benachrichtigung wenn die Maximale- bzw. Minimale-Temperaturgrenze über- bzw. unterschritten wird. Eventuell könnte in Zukunft auch eine Benachrichtigung an das Fahrzeug, beispielsweise zum Abholen eines Pakets direkt
aus der Web-Anwendung gesendet werden.


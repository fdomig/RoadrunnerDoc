\section{Transportüberwachung mittels Sensoren}\label{sensors}
\label{sec:sensors}

In diesem Projekt wurden die Transportüberwachung mittels Sensoren, welche
	von der Android Applikation (siehe Kapitel~\ref{sec:android})) überwacht
	werden, realisiert.
	
Für die in diesem Projekt spezifizierten Anforderungen (siehe Kapitel~\ref{sec:requirements})
	wurde die Temperatur sowie die aktuelle Position eines Gegenstands überwacht. Hierzu
	werden zwei unterschiedliche Sensortypen, welche in den beiden nachfolgenden Abschnitten
	erläutert werden, verwendet. Zusätzlich wurde die Zeitsynchronisation der mobilen Geräte
	mittels eigens entwickelter Synchronisierung (wie in Abschnitt~\ref{subsec:timesync}
	erläutert) realisiert.

\subsection{Temperaturüberwachung}

Temperatursensoren werden in diesem Projekt simuliert. Alle benötigten
	Temperatursensoren werden mit \emph{nodejs}, wie in
	Abschnitt~\ref{subsec:nodejs} erläutert, simuliert.

\subsection{Positionsüberwachung}

In diesem System werden in einem Zeitintervall von fünf Minuten die aktuelle
	Position eines auf dem Transportweg befinden Gegenstands aufgezeichnet.
	Die Positionsdaten werden von der Android Applikation bzw. der Service
	Applikation ausgelesen. Hierzu werden die Positionsdaten via GPS bzw. UMTS
	oder WLAN ermittelt und mit aufgezeichnet.
	
Die ermittelten Positionsdaten werden auf der Webapplikation bei den Lieferungen
	jeweils auf einer Karte dargestellt.
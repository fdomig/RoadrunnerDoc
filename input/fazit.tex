\section{Fazit}
\label{sec:fazit}

Die in diesem Projekt getroffene Entscheidung, ein ``Software Projekt''
	zu realisieren und keine eigene Hardware zu entwickeln war schlussendlich
	ausschlaggebend für einen erfolgreichen Projektabschluss.
	
Durch das Verwenden eines agilen Software-Enwicklungsprozesses konnte
	zu jedem Zeitpunkt ein lauffähiges Produkt gezeigt werden. Mit
	den Verwendeten Methoden des Extreme-Programming inklusive
	eines Test-Driven-Development Ansatzes konnten weitgehend alle
	Teile dieses Projekt ``fehlerfrei'' umgesetzt werden.
	
Mit einem neuen Ansatz der Datenpersistierung in einer
	Dokumenten-Orientierten-Datenbank wurde viel an Erfahrung gesammelt,
	sowie eine sehr interessante neue Technologie kennengelernt.
	
Mit dem Einsatz neuer Tools, Software, Frameworks und Applikationen
	wurde gezielt das Anti-Pattern ``Golden-Hammer'' \cite{Brown98}[S. 111]
	vermieden. Dieses besagt, ein Projekt mit den selben Tools und Systemen
	zu erstellen, welche sich bereits bewährt hat, führt dazu, dass oftmals
	die falsche Software eingesetzt wird, weil die Entwickler sich daran
	gewöhnt hat.
	
Leider mussten wir in diesem Projekt auch oftmals mit unfertigen Tools
	und Frameworks arbeiten, was manchmal auch Nachbesserung an diesen
	von Seiten des Projektteams benötigte. Jedoch hat das
	``Experimentieren'' und ``Ausprobieren'' neuer Software viel Spaß
	gemacht und interessante neue Ansätze aufgezeigt.
	
Die entwickelte Software wäre mit entsprechenden Erweiterungen durchwegs
	für ein reales Szenario zu verwenden und könnte in einem
	unternehmerischen Umfeld eingesetzt werden.
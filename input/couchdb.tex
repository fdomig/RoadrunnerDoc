\section{CouchDB}

\subsection{Dokumentstruktur}

\subsection{Designdokumente}

\subsection{Replizierung}

\subsection{MapReduce}

MapReduce ist ein Framework von Google, dass entwickelt wurde damit sehr große Datenmengen parallell bearbeitet werden können. CouchDB verwendet ebenfalls einen MapReduce-Anstaz um Daten aus der Datenbank zu lesen. Anhand eines Beispieles wird die Funktionsweise von MapReduce vorgestellt.

Das Beispiel beantwortet folgende Problemstellung: Welche Waren wurden gescannt und somit als geladen gekennzeichnet?

\subsubsection{Map - Phase}

Auf jedes Document in der Datenbank wird die Map-Methode angewendet. In einer Map-Methode werden Key-Value-Paarungen gebildet. Jedes Document in der Datenbank kann eine beliebige Anzahl an Key-Value-Paarungen generieren. Diese Key-Value-Paarungen werden in einem B-Tree gespeichert. Ändert sich nun ein Dokument müssen nur die entsprechenden Paarungen in dem B-Tree angepasst werden. 

\subsubsection{Reduce - Phase}

In der Reduce-Phase wird auf jeden Node in dem Tree die Reduce-Methode angewendet. Ziel der Reduce-Methode ist es die Datenmenge zu minimieren. Auf jede Node kann die Reduce-Methode beliebig oft angewendet werden. Daher wird die Reduce und die Rereduce-Phase unterschieden.

\subsection{CouchDB auf Android}

\subsection{CouchApp}


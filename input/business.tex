\section{Wirtschaftliche Betrachtung}
\label{sec:business}

In diesem Kapitel werden die finanziellen Aufwendungen für das Unternehmen, welches dieses System einsetzt, betrachtet.
Es werden die Kosten der Server-Infrastruktur, der Android Smartphones, der Sensoren und der Tarife beleuchtet.

\subsection{Infrastruktur für Server/Client}

Für das Backendsystem wird ein \emph{Apache} Webserver sowie ein \emph{CouchDB} Datenbankserver benötigt.
Bei einem kleinem System, bezogen auf die aktiv benutzten Smartphones und Transportmittel, können der Webserver und der Datenbankserver auf
einem physikalen Server installiert werden. Bei einem größeren System müssen diese beiden Server dementsprechend skaliert werden.
\\ \\
Der Preis für ein \emph{Rackmount} liegt zwischen € 700 bis € 1.000.

\subsection{Android Smartphones}

Der Kaufpreis für ein Android Smartphone liegt im Bereich von € 250 bis € 400 und jeder Fahrer benötigt sein eigenes Gerät.

\subsection{Bluetooth/Temperatur Sensoren}

Bluetooth/Temperatur Sensoren kosten zwischen € 150 und € 250 und jedes Transportmittel sollte mindestens über einen Sensor verfügen.

\subsection{Tarife}

Für Unternehmen gibt es inzwischen in den meisten Ländern Komplettpakete die einen Gesprächstarif sowie unlimitiertes Datenvolumen beinhalten.
Diese Komplettpakete kosten je nach Land und Umfang zwischen € 15 und € 80. Dabei gilt es zu beachten, dass sich diese Tarife nur auf das Inland beziehen.
Im Ausland fallen Roaming Kosten an die erheblich teurer sind, wobei es aber bei den meisten Netzbetreibern spezielle Roaming Tarife gibt.
\\ \\
Für das Deployment der mobilen Applikation und die Adminstration der Android Smartphones im Unternehmen wird 
\emph{Google Apps for Business}\footnote{vgl. \url{http://www.google.com/apps/intl/de/business/features.html}} benötigt.
Die jährlichen Kosten pro Smartphone belaufen sich auf € 40.



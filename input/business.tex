\section{Wirtschaftliche Betrachtung}
\label{sec:business}

In diesem Kapitel werden die finanziellen Aufwendungen für ein Unternehmen, welches
	dieses System einsetzt, betrachtet. Es werden die Kosten der Server-Infrastruktur,
	der Android Smartphones, der Sensoren und der Tarife betrachtet.

\subsection{Infrastruktur für Server/Client}

Für das Backendsystem wird ein \emph{Apache} Webserver sowie ein \emph{CouchDB}
	Datenbankserver benötigt. Bei einem kleinem System, bezogen auf die aktiv
	benutzten Smartphones und Transportmittel, können der Webserver und der
	Datenbankserver auf einem einzigen physikalischen Server installiert werden.
	Bei einem größeren Anwendungsszenario muss dieses System entsprechend
	skaliert werden. Der Preis für ein \emph{Rackmount} liegt derzeit zwischen
	700 bis 1.000 Euro.

\subsection{Android Smartphones}

Der Kaufpreis für ein Android Smartphone liegt aktuell im einem Bereich von 250
	bis 400 Euro. Jeder Benutzer (FahrerIn) benötigt ein eigenes Gerät.

\subsection{Temperatur Sensoren}

Temperatur Sensoren (Bluetooth) kosten derzeit zwischen 150 und 250 Euro.
	Jedes Transportmittel muss mindestens über einen Sensor verfügen.

\subsection{Tarife}

Für Unternehmen gibt es inzwischen in den meisten Ländern Komplettpakete die
	einen Gesprächstarif sowie unlimitiertes Datenvolumen beinhalten.
	
Diese Komplettpakete kosten je nach Land und Umfang zwischen 15 und 80 Euro.
	Dabei gilt es zu beachten, dass sich diese Tarife nur auf das Inland beziehen.
	Im Ausland fallen Roaming Kosten an, die erheblich teurer sind. Bei den meisten
	Netzbetreibern gibt es mittlerweile spezielle Roaming Tarife.

Für das Deployment der mobilen Applikation sowie die Administration der Android
	Smartphones im Unternehmen wird \emph{Google Apps for Business}\footnote{vgl.
	\url{http://www.google.com/apps/intl/de/business/features.html}} empfohlen.
	Die jährlichen Kosten pro Smartphone belaufen sich derzeit auf ca. 40 Euro.


